% \documentclass[dvipdfmx, a4paper, 11pt]{jsarticle}%A4用紙縦、明朝(デフォルト)11ポイント
\documentclass[dvipdfmx,titlepage, 11pt, a4paper]{jsarticle}%A4用紙縦、明朝(デフォルト)11ポイント
\usepackage[top=18truemm,bottom=18truemm,left=18truemm,right=18truemm]{geometry}%余白調整
% \setlength{\textheight}{45\baselineskip}
% \setlength{\textwidth}{46zw}% 46文字/行

\usepackage{template}

%%%====================================================================================================================================
\renewcommand{\thesection}{第\arabic{section}問}
\renewcommand{\thesubsection}{\thesection}
\titleformat*{\section}{\LARGE\mcfamily}%章のタイトルの文字の大きさを通常サイズに設定(明朝体で)
\titlespacing*{\section}{0pt}{*0}{0pt}%章番号の後の空白行の削除
\titleformat*{\subsection}{\Large\mcfamily}%節のタイトルの文字の大きさを通常サイズに設定(明朝体で)
\titlespacing*{\subsection}{0pt}{*0}{0pt}%節番号の後の空白行の削除
\titleformat*{\subsubsection}{\large\mcfamily}%小節のタイトルの文字の大きさを通常サイズに設定(明朝体で)
\titlespacing*{\subsubsection}{0pt}{*0}{0pt}%小節番号の後の空白行の削除

\makeatletter
%章番号付きコード番号
\AtBeginDocument{
  \renewcommand*{\thelstlisting}{\arabic{section}.\arabic{lstlisting}}%
  \@addtoreset{lstlisting}{section}
}

%章番号付き表番号
\renewcommand{\thetable}{%
    \arabic{section}.\arabic{table}%
}
\@addtoreset{table}{section}%

%章番号付き図番号
\renewcommand{\thefigure}{%
	\arabic{section}.\arabic{figure}%
}%
\@addtoreset{figure}{section}%

%章番号付き式番号
\renewcommand{\theequation}{%
	\arabic{section}.\arabic{equation}%
}%
\@addtoreset{equation}{section}%

\renewcommand{\p@enumiii}{}% 箇条書きの参照時の番号の変更(2(親の番号)a(子の番号)→aだけに)
\renewcommand{\p@enumii}{}% 箇条書きの参照時の番号の変更(2a→aだけに)
\makeatother
%%%====================================================================================================================================

%%%====================================================================================================================================
%%ページのレイアウト設定
\pagestyle{fancy}
\renewcommand{\sectionmark}[1]{\markboth{}{\thesection\ #1}}%                   %\rightmarkにセクション名を格納
\renewcommand{\subsectionmark}[1]{\markboth{#1}{\rightmark}}%   %\leftmarkにサブセクション名を格納
%[]は省略可で省略すると{}で指定された内容が偶数ページ奇数のどちらにも適用される。
% \renewcommand{\headrulewidth}{0pt} %ヘッダの罫線を消す 
\fancyfoot{}%                        %clear all footer fields
\lhead{\leftmark}%                   %左側ヘッダの定義[<偶数ページ>]{<奇数ページ>}
\chead{}%                            %中央ヘッダの定義[<偶数ページ>]{<奇数ページ>}
\rhead{\rightmark}%                  %右側ヘッダの定義[<偶数ページ>]{<奇数ページ>}
\lfoot{東工大数学h29年度}%       %左側フッターの定義[<偶数ページ>]{<奇数ページ>}
\cfoot{\thepage}%                    %中央フッターの定義[<偶数ページ>]{<奇数ページ>}
\rfoot{文殊の知恵}%                  %右側フッターの定義[<偶数ページ>]{<奇数ページ>}
\renewcommand{\headrulewidth}{0.1pt}%%ヘッダの線の太さ 
\renewcommand{\footrulewidth}{0.1pt}%%フッターの線の太さ
%%%====================================================================================================================================

\makeindex%索引用

\renewcommand{\refname}{}%参考文献の文字を非表示にする
\title{\Huge 東工大数学h29年度\\[10mm]}
\author{{\LARGE 文殊の知恵}\\[1mm]\LARGE \CID{8705}橋那弥}
\date{}

\begin{document}
\maketitle
\tableofcontents % 目次
\pagenumbering{roman}%目次のページ番号のスタイルをローマ数字にする
\newpage
\setcounter{tocdepth}{3}%章節の深さを3にするsubsubsectionまで
\pagenumbering{arabic}%他のページ番号は通常の数字にする。
\section{}%第1問
\subsection{問題文}
\begin{enumerate}[(1)]
  \item template
\end{enumerate}
\newpage
\subsection{解答}
\begin{enumerate}[(1)]
  \item 
  \begin{eqnarray*}
    \spaligndelims\vert\vert \spalignmat[c]{
      1 1 0 1;
      1 x 1 2;
      1 3 2 3;
      1 x^2 5 0
    } &=&
    \spaligndelims\vert\vert \spalignmat[c]{
      1 1 0 1;
      0 {x - 1} 1 1;
      0 2 2 2;
      0 {x^2 - 1} 5 {-1}
    }\\
    &=& (-1)^{1 + 1}\cdot 1
    \spaligndelims\vert\vert \spalignmat[c]{
      {x - 1} 1 1;
      2 2 2;
      {x^2 - 1} 5 {-1}
    }\\
    &=&
    \spaligndelims\vert\vert \spalignmat[c]{
      {x - 1} 1 1;
      {4 - 2x} 0 0;
      {x^2 + x - 2} 6 0
    }\\
    &=&(-1)^{1+3}\cdot 1
    \spaligndelims\vert\vert \spalignmat[c]{
      {4 - 2x} 0 ;
      {x^2 + x - 2} 6
    }\\
    &=& 12(2 - x)\\
    \therefore x = 2
  \end{eqnarray*}
  \item 
  \begin{enumerate}[(a)]
    \item 
    固有値$\lambda$を求める。
    \begin{eqnarray*}
      &&\spaligndelims\vert\vert \spalignmat[c]{
        A - \lambda I
      } = 0\\
      &\Longleftrightarrow&
      \spaligndelims\vert\vert \spalignmat[c]{
        {2 - \lambda} {2};
        {2} {5 - \lambda}
      } = 0\\
      &\Longleftrightarrow&
      4 - (2 - \lambda)(5 - \lambda) = 0\\
      &\Longleftrightarrow&
      -\lambda^2 + 7\lambda - 10 + 4 = 0\\
      &\Longleftrightarrow&
      (\lambda - 6)(\lambda - 1) = 0\\
      &&\therefore \lambda = 6, 1
    \end{eqnarray*}
    固有ベクトル空間を求める。
    \begin{eqnarray*}
      W(6;A) &=& \left\{\mathbold{v}|A\mathbold{v} = 6\mathbold{v}\right\}\\
      &=& \left\{\mathbold{v}|\mathbold{v} = \spalignmat[c]{
        {s};
        {2s}
      }\right\}\\
      W(1;A) &=& \left\{\mathbold{v}|A\mathbold{v} = \mathbold{v}\right\}\\
      &=& \left\{\mathbold{v}|\mathbold{v} = \spalignmat[c]{
        {-2t};
        {t}
      }\right\}
    \end{eqnarray*}
    \begin{eqnarray*}
      よって、\mathbold{v_1} = \spalignmat[c]{
        {1};
        {2}
        }, 
        \mathbold{v_2} = \spalignmat[c]{
          {-2};
          {1}
          }は独立なベクトルであるので、\mathbb{R}^2における基底となる。
          これより、正規直交基底ベクトル\mathbold{u_1}, \mathbold{u_2}を求めると、
    \end{eqnarray*}
    \begin{eqnarray*}
          \mathbold{u_1} &=& \frac{1}{|\mathbold{v_1}|}\mathbold{v_1} = \frac{1}{\sqrt{5}}\spalignmat[c]{
            {1};
            {2}
          }\\
          \mathbold{u_2}' &=& \mathbold{v_2} - \sum_{i = 1}^{1}(\mathbold{v_2}\cdot\mathbold{u_i})\mathbold{u_i}
          = \spalignmat[c]{
            {-2};
            {1}
          }\\
          \therefore \mathbold{u_2} &=& \frac{1}{|\mathbold{u_2}'|}\mathbold{u_2}'= \frac{1}{\sqrt{5}}\spalignmat[c]{
            {-2};
            {1}
          }
    \end{eqnarray*}
    よって、これらは$A$の固有ベクトルであり、正規直交基底となっているため、これらを並べた行列は$A$を対角かする正規直交行列$U$となる。
    \begin{equation*}
      U = \spalignmat[c]{
        {\mathbold{u_1}} {\mathbold{u_2}}
      } = \frac{1}{\sqrt{5}}\spalignmat[c]{
        {1} {-2};
        {2} {1}
      }
    \end{equation*}
    \item (a)より、$A$を対角化すると以下のようになり、$D$を以下のように定義する。
    \begin{equation*}
      U^{-1}AU = U^{\top}AU = \spalignmat[c]{
        {6} {0};
        {0} {1}
        } = D
    \end{equation*}
    よって、以下が成り立つ
    \begin{eqnarray*}
      D^n &=& U^{\top}A^{n}U\\
      A^{n} &=& UD^{n}U^{\top}\\
      A^n &=& \frac{1}{5}
      \spalignmat[c]{
        {1} {-2};
        {2} {1}
      }\spalignmat[c]{
        {6^n} {0};
        {0} {1}
      }\spalignmat[c]{
        {1} {2};
        {-2} {1}
      }\\
      A^n &=& \frac{1}{5}\spalignmat[c]{
        {6^n + 4} {2\cdot 6^n - 2};
        {2\cdot 6^n - 2} {4\cdot 6^n + 1}
      }
    \end{eqnarray*}
  \end{enumerate}
  \item 
  \begin{enumerate}[(a)]
    \item 点$P_0$を含む任意の平面の法線ベクトルを$\mathbold{n} = \spalignmat[c]{
      {a};
      {b};
      {c}
    }$とおくと、求める方程式は以下のようになる。\\
    この平面上にある任意の点$P(x, y, z)$を考えると、点$P_0$から点$P$へのベクトル$\mathbold{q} = \spalignmat[c]{
      {x - x_0};
      {y - y_0};
      {z - z_0}
    }$は法線ベクトルと直交する。よって、以下のようになる。
    \begin{eqnarray*}
      \mathbold{q}\cdot\mathbold{n} &=& 0\\
      a(x - x_0) + b(y - y_0) + c(z - z_0) &=& 0 \;\;(Eq.1)
    \end{eqnarray*}
    \item 方程式Eq.1が点$Q(x_1, y_1, z_1)$を含むとき方程式に代入して、以下のような式となる。
    \begin{eqnarray*}
      a(x_1 - x_0) + b(y_1 - y_0) + c(z_1 - z_0) &=& 0      
    \end{eqnarray*}
    \item 方程式Eq.1が題意の直線と平行であるとき、以下が成り立つ。\\
    実数定数$t$を用いて、直線の方程式から以下が成り立つ。
    \begin{eqnarray*}
      &&\begin{cases}
        \frac{x - x_1}{u} = t\\
        \frac{y - y_1}{v} = t\\
        \frac{z - z_1}{w} = t\\
      \end{cases}\\
      &\Longleftrightarrow&
      \begin{cases}
        x - x_1 = ut\\
        y - y_1 = vt\\
        z - z_1 = wt\\
      \end{cases}\\
      &\Longleftrightarrow&
      \spalignmat[c]{
        {x - x_1};
        {y - y_1};
        {z - z_1}
      } = t
      \spalignmat[c]{
        {u};
        {v};
        {w}
      }
    \end{eqnarray*}
    よって、この直線は方向ベクトル$\mathbold{p} = 
          \spalignmat[c]{
            {u};
            {v};
            {w}
          }$で表される直線である。\\
    従って、平行であるときは法線ベクトルと方向ベクトルが直交するときであるので、以下が成り立つ。
    \begin{eqnarray*}
      \mathbold{n}\cdot\mathbold{p} &=& 0\\
      au + bv + cw &=& 0
    \end{eqnarray*}
    よって、平面と直線が平行であることを表す式は以下のようになる。
    \begin{equation*}
      au + bv + cw = 0
    \end{equation*}
    \item 題意の式から以下が成り立つ。
    \begin{eqnarray}
      &&\spaligndelims\vert\vert \spalignmat[c]{
        {x - x_0} {y - y_0} {z - z_0};
        {x_1 - x_0} {y_1 - y_0} {z_1 - z_0};
        u v w
      } = 0 \nonumber\\
      &\Longleftrightarrow&
      (x - x_0)
      \spaligndelims\vert\vert \spalignmat[c]{
        {y_1 - y_0} {z_1 - z_0};
        v w
      }
      - (y - y_0)
      \spaligndelims\vert\vert \spalignmat[c]{
        {x_1 - x_0} {z_1 - z_0};
        u w
      }
      + (z - z_0)
      \spaligndelims\vert\vert \spalignmat[c]{
        {x_1 - x_0} {y_1 - y_0};
        u v
      }
      = 0\nonumber\\
      &\Longleftrightarrow&
      (x - x_0)\{(y_1 - y_0)w - (z_1 - z_0)v\}
      - (y - y_0)\{(x_1 - x_0)w - (z_1 - z_0)u\}\nonumber\\ 
      &&+ (z - z_0)\{(x_1 - x_0)v - (y_1 - y_0)u\}
      = 0\label{pro1:sub3:eq1}
    \end{eqnarray}
    求める方程式は方程式Eq.1を満たし、上記の条件と点$Q(x_1, y_1, z_1)$を含むことを満たすため、
    (a), (b), (c)より以下が成り立つ。\\
    \begin{eqnarray}
      au + bv + cw &=& 0 \label{pro1:sub3:subd:eq1}\\
      a(x - x_0) + b(y - y_0) + c(z - z_0) &=& 0 \;\; (Eq.1)\nonumber\\
      a(x_1 - x_0) + b(y_1 - y_0) + c(z_1 - z_0) &=& 0 \label{pro1:sub3:subd:eq2}
    \end{eqnarray}
    また、題意より、$u, v, w \neq 0$である。これと\eqref{pro1:sub3:subd:eq1}より、以下が成り立つ。
    \begin{equation*}
      \begin{cases}
        a = -\frac{bv + cw}{u}\\
        b = -\frac{au + cw}{v}\\
      \end{cases}
    \end{equation*}
    よって、これと、\eqref{pro1:sub3:subd:eq2}より、以下が成り立つ。
    \begin{eqnarray*}
      &&\begin{cases}
        -\frac{bv + cw}{u}(x_1 - x_0) + b(y_1 - y_0) + c(z_1 - z_0) = 0\\
        a(x_1 - x_0) - \frac{au + cw}{v}(y_1 - y_0) + c(z_1 - z_0) = 0\\
      \end{cases}\\
      &\Longleftrightarrow&
      \begin{cases}
        -bv(x_1 - x_0) + bu(y_1 - y_0) -cw(x_1 - x_0) + cu(z_1 - z_0) = 0\\
        av(x_1 - x_0) - au(y_1 - y_0) -cw(y_1 - y_0) + cv(z_1 - z_0) = 0\\
      \end{cases}\\
      &\Longleftrightarrow&
      \begin{cases}
        b\{u(y_1 - y_0) - v(x_1 - x_0)\} = c\{w(x_1 - x_0) - u(z_1 - z_0)\}\\
        a\{v(x_1 - x_0) - u(y_1 - y_0)\} = c\{w(y_1 - y_0) - v(z_1 - z_0)\}\\
      \end{cases}
    \end{eqnarray*}
    よって、$s = v(x_1 - x_0) - u(y_1 - y_0)$とおくと以下が成り立つ。
    \begin{eqnarray*}
      \begin{cases}
        bs = -c\{w(x_1 - x_0) - u(z_1 - z_0)\}\\
        as = c\{w(y_1 - y_0) - v(z_1 - z_0)\}\\
      \end{cases}
    \end{eqnarray*}
    よって、方程式Eq.1より以下が成り立つ。
    \begin{eqnarray*}
      &&as(x - x_0) + bs(y - y_0) + cs(z - z_0) = 0\\
      &\Longleftrightarrow&
      c\{w(y_1 - y_0) - v(z_1 - z_0)\}(x - x_0) - c\{w(x_1 - x_0) - u(z_1 - z_0)\}(y - y_0) + cs(z - z_0) = 0\\
      &\Longleftrightarrow&
      \{w(y_1 - y_0) - v(z_1 - z_0)\}(x - x_0) - \{w(x_1 - x_0) - u(z_1 - z_0)\}(y - y_0) + s(z - z_0) = 0\\
      &\Longleftrightarrow&
      \{w(y_1 - y_0) - v(z_1 - z_0)\}(x - x_0) + \{u(z_1 - z_0) - w(x_1 - x_0)\}(y - y_0) + \{v(x_1 - x_0) - u(y_1 - y_0)\}(z - z_0) = 0\\
    \end{eqnarray*}
    よって、これは式\eqref{pro1:sub3:eq1}と一致する。
    従って、題意は示された。
  \end{enumerate}
  \item 題意の式より以下が成り立つ。
  \begin{eqnarray*}
    Q(\mathbold{x}) &=& \sum_{i = 1}^{n}\sum_{j = 1}^{n}h_{ij}x_{i}x_{j}
    = \mathbold{x}^{\top}H\mathbold{x}\\
    \mathbold{x}^{\top}\mathbold{x} &=& \sum_{i = 1}^{n}x_{i}{}^2 = 1
  \end{eqnarray*}
  ここで$H$の固有値を$\lambda_{i}\;\;(1 \leq i \leq n)$とおくと、それに対応する固有ベクトル$\mathbold{v_i}$が存在し、以下が成り立つ。
  \begin{eqnarray*}
    H\mathbold{v_i} = \lambda_{i}\mathbold{v_i}
  \end{eqnarray*}
  よって、この時、$\mathbold{v_i}$はそれぞれ独立であり、$\mathbb{R}$の基底となる。よって、これの正規直交基底も存在し、それらもまた$H$の固有ベクトルとなる。
  その正規直交ベクトルを$\mathbold{u_i}$とおき、それらを並べた行列を$U$とおくと以下が成り立つ。
  \begin{eqnarray*}
    U = \spalignmat[c]{
      {\mathbold{u_1}} {\mathbold{u_2}} {\cdots} {\mathbold{u_n}}
    }
  \end{eqnarray*}
  よって、この時$U$は$H$を対角化する直交行列となるので以下が成り立つ。
  \begin{eqnarray*}
    U^{-1}HU = U^{\top}HU = D = \spalignmat[c]{
      {\lambda_1} {0} {\cdots} {0};
      {0} {\lambda_2} {\cdots} {0};
      {\vdots} {\vdots} {\ddots} {\vdots};
      {0} {0} {\cdots} {\lambda_n}
    }
  \end{eqnarray*}
  よって、$\mathbold{x} = U\mathbold{y}$とおくと、以下が成り立つ。
  \begin{eqnarray*}
    \mathbold{x}^{\top}\mathbold{x} &=& \mathbold{y}^{\top}U^{\top}U\mathbold{y} = \mathbold{y}^{\top}\mathbold{y} = 1\\
    Q(\mathbold{x}) &=& \left(U\mathbold{y}\right)^{\top}H\left(U\mathbold{y}\right) = \mathbold{y}^{\top}U^{\top}HU\mathbold{y} 
    = \mathbold{y}^{\top}D\mathbold{y}\\
    &=& \frac{\mathbold{y}^{\top}D\mathbold{y}}{\mathbold{y}^{\top}\mathbold{y}} = \frac{\sum_{i = 1}^{n}\lambda_{i}y_{i}^2}{\sum_{i = 1}^{n}y_{i}^2}
  \end{eqnarray*}
  よって、加重平均より、この式の最大値は$\lambda_{i}$の最大値すなわち固有値の最大値と一致することが示された。
\end{enumerate}
\newpage
\section{}
\subsection{問題文}
\newpage
\subsection{解答}
\begin{enumerate}[(1)]
  \item $A_1$の状態遷移関数は以下のようになる。
  \begin{table}[htbp]
    \centering
    \begin{tabular}{c|cc}
      \hline
      $\delta$ & $0$ & $1$\\
      \hline
      $q_0$ & $q_0$ & $q_1$\\
      $q_1$ & $q_0$ & $q_1$\\
      \hline
    \end{tabular}
  \end{table}
  \item $A_1$の終了状態を$F=\{q_0\}$から$F=\{q_1\}$に変えればいい。
  \item $A_2$の入力記号$\Sigma$と状態集合と初期状態は$Q_2 = \{r_0, r_1, r_2\}, q_{init} = r_0$と与えられているため、状態遷移関数と終了状態を定義する。
  状態をそれぞれ$3$で割ったあまりで分けるとし、$r_i$を余り$i\;\;(1\leq i \leq 2)$とおくと以下のように遷移できる。
  \begin{itemize}
    \item [$r_0$] 
    \begin{itemize}
      \item[$0$] 
      $2$倍を意味するので、その値は余り$0$のままより状態は変化しない。
      \item[$1$] 
      $2$倍して$1$を足すことを意味するので、$2(3k) + 1 = 3\cdot 2k + 1$よりあまり$1$より状態$r_1$へ。
    \end{itemize}
    \item [$r_1$] 
    \begin{itemize}
      \item[$0$] 
      $2$倍を意味する。$2(3k + 1) = 3\cdot 2k + 2$より余り$2$よって、状態$r_2$へ。
      \item[$1$] 
      $2$倍して$1$を足すことを意味するので、$2(3k + 1) + 1 = 3(2k + 1)$より余り$0$より状態$r_0$へ。
    \end{itemize}
    \item [$r_2$] 
    \begin{itemize}
      \item[$0$] 
      $2$倍を意味する。$2(3k + 2) = 3(2k + 1) + 1$より余り$1$よって、状態$r_1$へ。
      \item[$1$] 
      $2$倍して$1$を足すことを意味するので、$2(3k + 2) + 1 = 3(2k + 1) + 2$より余り$2$より状態$r_2$へ。
    \end{itemize}
  \end{itemize}
  よって状態遷移表は以下のようになる。
  \begin{table}[htbp]
    \centering
    \begin{tabular}{c|cc}
      \hline
      $\delta$ & $0$ & $1$\\
      \hline
      $r_0$ & $r_0$ & $r_1$\\
      $r_1$ & $r_2$ & $r_0$\\
      $r_2$ & $r_1$ & $r_2$\\
      \hline
    \end{tabular}
  \end{table}
  終了状態は$F_2 = {r_0}$
  \item $A_1$と$A_2$の積を取ると以下のように$A_3$は定義できる。
  状態集合$Q_3$は
  \begin{eqnarray*}
    Q_3 &=& Q_1 \times Q_2\\
    &=& \{(q_0, r_0), (q_0, r_1), (q_0, r_2), (q_1, r_0), (q_1, r_1), (q_1, r_2)\}
  \end{eqnarray*}
  初期状態は$q_{init} = (q_0, r_0)$
  入力記号は$\Sigma$、 終了状態は$F_3 = (q_0, r_0)$
  状態遷移関数は以下のようになる。
  \begin{table}[htbp]
    \centering
    \begin{tabular}{c|cc}
      \hline
      $\delta$ & $0$ & $1$\\
      \hline
      $(q_0, r_0)$ & $(q_0, r_0)$ & $(q_1, r_1)$\\
      $(q_0, r_1)$ & $(q_0, r_2)$ & $(q_1, r_0)$\\
      $(q_0, r_2)$ & $(q_0, r_1)$ & $(q_1, r_2)$\\
      $(q_1, r_0)$ & $(q_0, r_0)$ & $(q_1, r_1)$\\
      $(q_1, r_1)$ & $(q_0, r_2)$ & $(q_1, r_0)$\\
      $(q_1, r_2)$ & $(q_0, r_1)$ & $(q_1, r_2)$\\
      \hline
    \end{tabular}
  \end{table}
  $A_3$の状態集合は$A_1$と$A_2$の状態集合の積集合であり、終了状態も$A_1$の終了状態かつ$A_2$の終了状
  態となる状態を終了状態とし、状態遷移関数は$A_1$と$A_2$の状態遷移関数をそれぞれ適応したものとなっている。
  \item $A_1$と$A_2$の和を取った非決定性オートマトン$A_4$は以下のようになる。
  状態集合$Q_4 = Q_3$よって、初期状態$q_{init} = s_0 = (q_0, r_0)$
  終了状態は和より$F_4 = \{q_0\}\times Q_2 \cup \{r_0\} \times Q_1 = \{(q_0, r_0), (q_0, r_1), (q_0, r_2), (q_1, r_0)\}$
  招待遷移関数は非決定性より(4)から以下のようになる。
  \begin{table}[H]
    \centering
    \begin{tabular}{c|cc}
      \hline
      $\delta$ & $0$ & $1$\\
      \hline
      $(q_0, r_0)$ & $\{(q_0, r_0)\}$ & $\{(q_1, r_1)\}$\\
      $(q_0, r_1)$ & $\{(q_0, r_2)\}$ & $\{(q_1, r_0)\}$\\
      $(q_0, r_2)$ & $\{(q_0, r_1)\}$ & $\{(q_1, r_2)\}$\\
      $(q_1, r_0)$ & $\{(q_0, r_0)\}$ & $\{(q_1, r_1)\}$\\
      $(q_1, r_1)$ & $\{(q_0, r_2)\}$ & $\{(q_1, r_0)\}$\\
      $(q_1, r_2)$ & $\{(q_0, r_1)\}$ & $\{(q_1, r_2)\}$\\
      \hline
    \end{tabular}
  \end{table}
  $A_3$の状態集合は$A_1$と$A_2$の状態集合の積集合であり、終了状態は$A_1$または$A_2$をみたせばいいので、どちらかの終了状態が
  含まれる集合とし、状態遷移関数は$A_1$と$A_2$の状態遷移関数をそれぞれ適応し、適応した先は集合としている。
  \item $A_5$は$A_4$と状態遷移関数以外は全て同じになり、状態遷移関数は$A_3$と同様になる。
\end{enumerate}
\newpage
\section{}
\subsection{問題文}
\newpage
\subsection{解答}
\begin{enumerate}[(1)]
  \item \begin{enumerate}[(a)]
    \item 0
    \item 以下のようになる。
    \begin{figure}[htbp]
      \centering
      \begin{tikzpicture}[>=stealth]
        \coordinate (O) at (0, 0) node [] {$f(0, 0)$};
        \draw [->] (O)++(-170:0.5cm) -- ++(-170:3cm) node (A) [] {};
        \node (A) at ($(O)!1.3!(A)$) {$f(0, 1)$};
        \draw [->] (O)++(-10:0.5cm) -- ++(-10:3cm) node (B) [] {};
        \node (B) at ($(O)!1.3!(B)$) {$f(8, 1)$};
        \draw [->] (A)++(-45:0.5cm) -- ++(-45:1.7cm) node (C) [] {};
        \node (C) at ($(A)!1.3!(C)$) {$f(2, 2)$};
        \draw [->] (A)++(-135:0.5cm) -- ++(-135:1.7cm) node (D) [] {};
        \node (D) at ($(A)!1.3!(D)$) {$f(0, 2)$};
        \draw [->] (D)++(-135:0.5cm) -- ++(-135:1cm) node (E) [] {};
        \node (E) at ($(D)!1.3!(E)$) {$f(0, 3)$};
        \draw [->] (D)++(-45:0.5cm) -- ++(-45:1cm) node (F) [] {};
        \node (F) at ($(D)!1.3!(F)$) {$f(4, 3)$};
        \draw [->] (C)++(-45:0.5cm) -- ++(-45:1cm) node (G) [] {};
        \node (G) at ($(C)!1.3!(G)$) {$f(6, 3)$};
        \draw [->] (C)++(-135:0.5cm) -- ++(-135:1cm) node (H) [] {};
        \node (H) at ($(C)!1.3!(H)$) {$f(2, 3)$};
        \draw [->] (B)++(-45:0.5cm) -- ++(-45:1.7cm) node (I) [] {};
        \node (I) at ($(B)!1.3!(I)$) {$f(10, 2)$};
        \draw [->] (B)++(-135:0.5cm) -- ++(-135:1.7cm) node (J) [] {};
        \node (J) at ($(B)!1.3!(J)$) {$f(8, 2)$};
        \draw [->] (J)++(-45:0.5cm) -- ++(-45:1cm) node (K) [] {};
        \node (K) at ($(J)!1.3!(K)$) {$f(12, 3)$};
        \draw [->] (J)++(-135:0.5cm) -- ++(-135:1cm) node (L) [] {};
        \node (L) at ($(J)!1.3!(L)$) {$f(8, 3)$};
        \draw [->] (I)++(-45:0.5cm) -- ++(-45:1cm) node (M) [] {};
        \node (M) at ($(I)!1.3!(M)$) {$f(14, 3)$};
        \draw [->] (I)++(-135:0.5cm) -- ++(-135:1cm) node (N) [] {};
        \node (N) at ($(I)!1.3!(N)$) {$f(10, 3)$};
      \end{tikzpicture}
    \end{figure}
    \item $s$は$A$の部分集合$S$の総和$s = \sum_{a\in S}a$を表し、$i$は
    部分集合$S$に次に入れるべき要素の配列番号を表す。
  \end{enumerate}
  \item \begin{enumerate}[(a)]
    \item s
    \item 以下のようになる。
    \begin{figure}[htbp]
      \centering
      \begin{tikzpicture}[>=stealth]
        \coordinate (O) at (0, 0) node [] {$f(0, 0)$};
        \draw [->] (O)++(-135:0.5cm) -- ++(-135:1cm) node (A) [] {};
        \node (A) at ($(O)!1.3!(A)$) {$f(0, 1)$};
        \draw [->] (O)++(-45:0.5cm) -- ++(-45:1cm) node (B) [] {};
        \node (B) at ($(O)!1.3!(B)$) {$f(8, 1)$};
        \draw [->] (A)++(-45:0.5cm) -- ++(-45:1.7cm) node (C) [] {};
        \node (C) at ($(A)!1.3!(C)$) {$f(2, 2)$};
        \draw [->] (A)++(-135:0.5cm) -- ++(-135:1.7cm) node (D) [] {};
        \node (D) at ($(A)!1.3!(D)$) {$f(0, 2)$};
        \draw [->] (D)++(-135:0.5cm) -- ++(-135:1cm) node (E) [] {};
        \node (E) at ($(D)!1.3!(E)$) {$f(0, 3)$};
        \draw [->] (D)++(-45:0.5cm) -- ++(-45:1cm) node (F) [] {};
        \node (F) at ($(D)!1.3!(F)$) {$f(4, 3)$};
        \draw [->] (C)++(-45:0.5cm) -- ++(-45:1cm) node (G) [] {};
        \node (G) at ($(C)!1.3!(G)$) {$f(6, 3)$};
        \draw [->] (C)++(-135:0.5cm) -- ++(-135:1cm) node (H) [] {};
        \node (H) at ($(C)!1.3!(H)$) {$f(2, 3)$};
      \end{tikzpicture}
    \end{figure}
    \item 関数の呼び出し回数が減少し、実行時間が改善された。この関数内での計算は
    部分集合の総和を計算し、要素は全て正の整数であるため、値が減少することはない。よって、
    総和を計算していくうちに目的の数字より大きくなってしまった場合はそれ以上足しても一致することはないため、
    その部分を省略することができるため。
  \end{enumerate}
  \item \begin{enumerate}[(a)]
    \item $a_{n - 1}$\\
    成り立つ理由は以下が成り立つためである。
    命題$P$を$\left(\sum_{a\in T} a\right) = k$となる$A$の部分集合$T(T\subseteq A)$が存在することとおき、
    命題$Q$を$\left(\sum_{a\in U} a\right) = k$となる$A'$の部分集合$U(U\subseteq A')$が存在するまたは、
    $\left(\sum_{a\in V}a \right) = k - a_{n - 1}$となる$A'$の部分集合$V(V\subseteq A')$が存在することとおく。
    この時$P\Longleftrightarrow Q$が成り立つことを示す。\\
    $P \Rightarrow Q$を示す。\\
    $P$より$T$について以下の場合分けが考えられる。
    \begin{enumerate}[(i)]
      \item $a_{n - 1} \in T$
      \item $a_{n - 1} \notin T$
    \end{enumerate}
    それぞれについて$Q$が成り立つことを示す。
    \begin{enumerate}[(i)]
      \item $a_{n - 1} \in T, a_{n - 1} \notin A'$より$U$は存在しない。\\
      $T' = T - \{a_{n - 1}\}$とおくと、 $T' \subseteq A'$となる。\\
      この時、$T'$の総和は$k - a_{n - 1}$である。よって、$V=T'$は存在する。
      \item $a_{n - 1} \notin T$より$T\subseteq A'$となる。\\
      従って、$U = T$となる$U$は存在する。
    \end{enumerate}
    よって、$P\Rightarrow Q$が示された。\\
    $Q \Rightarrow P$を示す。\\
    $Q$より以下の場合分けが考えられる。
    \begin{enumerate}[(i)]
      \item $U$が存在する。
      \item $V$が存在する。
    \end{enumerate}
    それぞれについて$P$が成り立つことを示す。
    \begin{enumerate}[(i)]
      \item $U$が存在するとき、$U\subseteq A'\subset A$より$U$は$A$の部分集合であり、
      $(\sum_{a\in U}a) = k$より、条件を満たす。
      \item $V$が存在するとき、$V' = V \cup \{a_{n - 1}\}$とおくと、$V \subseteq A' \subset A, a_{n - 1}\in A$より$V' \subseteq A$。
      また、$(\sum_{a\in V}a) = k - a_{n - 1}$より、$(\sum_{a\in V'}a) = k - a_{n - 1} + a_{n - 1} = k$となる。よって、この時も条件を満たす。
    \end{enumerate}
    従って、$Q\Rightarrow P$も示された。\\
    以上より、$P\Longleftrightarrow Q$が成り立つ。
    \item \begin{enumerate}
      \item[\maru{3}{black}] $i == 0 \;\;\&\& \;\;j == a[i]$
      \item[\maru{4}{black}] $j$
      \item[\maru{5}{black}] $j - a[i]$  
    \end{enumerate}
    \item 配列$b$の値は以下のようになる。
    \begin{table}[htbp]
      \centering
      \begin{tabular}{|c|c|c|c|c|c|c|c|}
        \hline
        & $0$ & $1$ & $2$ & $3$ & $4$ & $5$ & $6$\\
        \hline
        $0$ & $1$ & $0$ & $0$ & $0$ & $0$ & $0$ & $0$\\
        \hline
        $1$ & $1$ & $0$ & $1$ & $0$ & $0$ & $0$ & $0$\\
        \hline
        $2$ & $1$ & $0$ & $1$ & $0$ & $1$ & $0$ & $1$\\
        \hline
      \end{tabular}
    \end{table}
    \item 集合$A$における要素数$i + 1$の部分集合$A'\subseteq A$において、$(\sum_{a\in W} a) = k$
    となる部分集合$W\subseteq A'$が存在することを意味する。
    \item 図3.1のプログラムのオーダーは$O(2^n)$であり、図3.2のプログラムのオーダーは$O(nk)$より
    計算時間の観点においては図3.2のほうが優れている。
  \end{enumerate}
\end{enumerate}
\newpage
\section{}
\subsection{問題文}
\newpage
\subsection{解答}
\begin{enumerate}[(1)]
  \item \begin{enumerate}[(a)]
    \item 
    \begin{eqnarray*}
      a_0 &=&\frac{1}{\pi}\dint{-\pi}{\pi}{f_1(x)}\\
      &=& \frac{1}{\pi}\biggl[\frac{x^2}{2}\biggr]_{-\pi}^{\pi}\\
      &=& 0
    \end{eqnarray*}
    \begin{eqnarray*}
      a_k &=&\frac{1}{\pi}\dint{-\pi}{\pi}{f_1(x)\cos kx}\\
      &=&\frac{1}{\pi}\biggl[x\frac{\sin kx}{k}\biggr]_{-\pi}^{\pi} - \frac{1}{k\pi}\dint{-\pi}{\pi}{\sin kx}\\
      &=&-\frac{1}{k\pi}\biggl[\frac{-\cos kx}{k}\biggr]_{-\pi}^{\pi} = 0
    \end{eqnarray*}
    \begin{eqnarray*}
      b_k &=&\frac{1}{\pi}\dint{-\pi}{\pi}{f_1(x)\sin kx}\\
      &=&-\frac{1}{\pi}\biggl[x\frac{\cos kx}{k}\biggr]_{-\pi}^{\pi} + \frac{1}{k\pi}\dint{-\pi}{\pi}{\cos kx}\\
      &=&\frac{1}{k\pi}\biggl[\frac{\sin kx}{k}\biggr]_{-\pi}^{\pi} - \frac{\pi}{k\pi}\{\cos(k\pi) + \cos(-k\pi)\}\\
      &=&\frac{2\cdot (-1)^{k + 1}}{k}
    \end{eqnarray*}
    \item 
    \begin{eqnarray*}
      a_0 &=&\frac{1}{\pi}\dint{-\pi}{\pi}{f_2(x)}\\
      &=& \frac{1}{\pi}\left\{\dint{-\pi}{-\frac{\pi}{2}}{-1} + \dint{-\frac{\pi}{2}}{\frac{\pi}{2}}{1} + 
      \dint{\frac{\pi}{2}}{\pi}{-1}\right\}\\
      &=& \frac{1}{\pi}\left(-\frac{\pi}{2} + \pi - \frac{\pi}{2}\right) = 0
    \end{eqnarray*}
    \begin{eqnarray*}
      a_k &=&\frac{1}{\pi}\dint{\pi}{-\pi}{f_2(x)\cos kx}\\
      &=& \frac{1}{\pi}\left\{\dint{-\pi}{-\frac{\pi}{2}}{-\cos kx} + \dint{-\frac{\pi}{2}}{\frac{\pi}{2}}{\cos kx} + 
      \dint{\frac{\pi}{2}}{\pi}{-\cos kx}\right\}\\
      &=& \frac{1}{\pi}\left\{-\biggl[\frac{\sin kx}{k}\biggr]_{-\pi}^{-\frac{\pi}{2}} + \biggl[\frac{\sin kx}{k}\biggr]_{-\frac{\pi}{2}}^{\frac{\pi}{2}}
      - \biggl[\frac{\sin kx}{k}\biggr]_{\frac{\pi}{2}}^{\pi}\right\}\\
      &=& \frac{1}{k\pi}\left\{\sin \left(\frac{k\pi}{2}\right) + \sin\left(\frac{k\pi}{2}\right) + \sin\left(\frac{k\pi}{2}\right) + \sin\left(\frac{k\pi}{2}\right)\right\}\\
      &=& \frac{4\sin\left(\frac{k\pi}{2}\right)}{k\pi}
    \end{eqnarray*}
    \begin{eqnarray*}
      b_k &=&\frac{1}{\pi}\dint{-\pi}{\pi}{f_2(x)\sin kx}\\
      &=& \frac{1}{\pi}\left\{\dint{-\pi}{-\frac{\pi}{2}}{-\sin kx} + \dint{-\frac{\pi}{2}}{\frac{\pi}{2}}{\sin kx} + 
      \dint{\frac{\pi}{2}}{\pi}{-\sin kx}\right\}\\
      &=& \frac{1}{\pi}\left\{-\biggl[\frac{\cos kx}{k}\biggr]_{-\pi}^{-\frac{\pi}{2}} + \biggl[\frac{\cos kx}{k}\biggr]_{-\frac{\pi}{2}}^{\frac{\pi}{2}}
      - \biggl[\frac{\cos kx}{k}\biggr]_{\frac{\pi}{2}}^{\pi}\right\} = 0
    \end{eqnarray*}
    \item \begin{equation*}
      (左辺) = \frac{1}{\pi}\dint{-\pi}{\pi}{|f(x)^2|}
    \end{equation*}
    ここで、
    \begin{eqnarray*}
      f(x)^2 &=& \frac{a_0^2}{4} + a_0\sum_{k = 1}^{\infty}(a_k\cos kx + b_k\sin kx) 
      + \left\{\sum_{k = 1}^{\infty}(a_k\cos kx + b_k\sin kx)\right\}\left\{\sum_{l = 1}^{\infty}(a_l\cos lx + b_l\sin lx)\right\}\\
      &=& \frac{a_0^2}{4} + a_0\sum_{k = 1}^{\infty}(a_k\cos kx + b_k\sin kx)\\
      &&+ \sum_{k = 1}^{\infty}\sum_{l = 1}^{\infty}(a_ka_l\cos kx\cos lx + b_kb_l\sin kx\sin lx + a_kb_l\cos kx\sin lx + b_ka_l\sin kx\cos lx)
    \end{eqnarray*}
    よって、
    \begin{eqnarray*}
      (左辺) &=& \frac{a_0^2}{4\pi}\dint{-\pi}{\pi}{1} + \frac{a_0}{\pi}\sum_{k = 1}^{\infty}\dint{-\pi}{\pi}{(a_k\cos kx + b_k\sin kx)}\\
      && + \frac{1}{\pi}\sum_{k = 1}^{\infty}\sum_{l = 1}^{\infty}\dint{-\pi}{\pi}{(a_ka_l\cos kx\cos lx + b_kb_l\sin kx\sin lx + a_kb_l\cos kx\sin lx + b_ka_l\sin kx\cos lx)}\\
      &=& \frac{a_0^2}{2} + \frac{a_0}{\pi}\sum_{k = 1}^{\infty}I_{k} + \frac{1}{\pi}\sum_{k = 1}^{\infty}\sum_{l = 1}^{\infty}J_{kl}
    \end{eqnarray*}
    上記のように$I_{k}, J_{kl}$をおく。
    この時以下のようになる。
    \begin{eqnarray*}
      I_{k} &=& a_k\biggl[\frac{\sin kx}{k}\biggr]_{-\pi}^{\pi} + b_k\biggl[\frac{-\cos kx}{k}\biggr]_{-\pi}^{\pi} = 0\\
      J_{kl} &=& a_ka_l\dint{-\pi}{\pi}{\left\{\frac{\cos (k + l)x + \cos(k - l)x}{2}\right\}} - b_kb_l\dint{-\pi}{\pi}{\left\{\frac{\cos (k + l)x - \cos(k - l)x}{2}\right\}}\\
      && + a_kb_l\dint{-\pi}{\pi}{\left\{\frac{\sin (k + l)x - \sin(k - l)x}{2}\right\}} + b_ka_l\dint{-\pi}{\pi}{\left\{\frac{\sin (k + l)x + \sin(k - l)x}{2}\right\}}
    \end{eqnarray*}
    \begin{enumerate}[(i)]
      \item $k = l$の時、
      \begin{eqnarray*}
        J_{kk} &=& a_k^2\dint{-\pi}{\pi}{\left\{\frac{\cos 2kx + 1}{2}\right\}}
        - b_k^2\dint{-\pi}{\pi}{\left\{\frac{\cos 2kx - 1}{2}\right\}}\\
        && + a_kb_k\dint{-\pi}{\pi}{\left\{\frac{\sin 2kx}{2}\right\}} + b_ka_k\dint{-\pi}{\pi}{\left\{\frac{\sin 2kx}{2}\right\}}\\
        &=& \frac{a_k^2}{2}\dint{-\pi}{\pi}{1} -\frac{b_k^2}{2}\dint{-\pi}{\pi}{-1} = \pi(a_k^2 + b_k^2)
      \end{eqnarray*}
      \item $k \neq l$の時、
      \begin{eqnarray*}
        J_{kl} &=& 0  
      \end{eqnarray*}
    \end{enumerate}
    従って、
    \begin{eqnarray*}
      (左辺) &=& \frac{a_0^2}{2} + 0 + \frac{1}{\pi}\sum_{k = 1}^{\infty}\pi(a_k^2 + b_k^2)\\
      &=& \frac{a_0^2}{2} + \sum_{k = 1}^{\infty}(a_k^2 + b_k^2) = (右辺)
    \end{eqnarray*}
    よって示された。
  \end{enumerate}
  \item 
  \begin{enumerate}[(a)]
    \item フーリエ変換すると
    \begin{eqnarray*}
      G(\omega) &=& \dint[t]{-\infty}{\infty}{g(t)e^{-i\omega t}}\\
      &=& \sum_{k = -\infty}^{\infty}\dint[t]{-\infty}{\infty}{\delta(t - kT)e^{-i\omega t}}\\
      &=& \sum_{k = -\infty}^{\infty}e^{-i\omega kT}\\
      &=& 2\pi\sum_{k = -\infty}^{\infty}\delta(-T\omega - 2\pi k)\\
      &=& \frac{2\pi}{T}\sum_{k = -\infty}^{\infty}\delta\left(\omega - \frac{-2\pi k}{T}\right)\\
      &&\therefore \Omega = -\frac{2\pi}{T}
    \end{eqnarray*}
    \item 与式より
    \begin{eqnarray*}
      X_g(\omega) &=& \frac{1}{2\pi}\dint[\omega']{-\infty}{\infty}{X(\omega - \omega')G(\omega')}\\
      &=& \frac{2\pi}{2T\pi}\sum_{k = -\infty}^{\infty}\dint[\omega']{-\infty}{\infty}{X(\omega - \omega')\delta(\omega' - k\Omega)}\\
      &=& \frac{1}{T}\sum_{k = -\infty}^{\infty}X(\omega - k\Omega)\\
      &=& \frac{1}{T}\sum_{k = -\infty}^{\infty}X\left(\omega + \frac{2\pi k}{T}\right)\\
    \end{eqnarray*}
    \item $X_g(\omega)$の概形は以下のようになる。
    \begin{figure}[htbp]
      \centering
      \begin{tikzpicture}[>=stealth]
        \coordinate (O) at (0, 4.7) node at (O) [below] {$0$};
        \coordinate (A) at ($(O) + (-6, 0)$);
        \coordinate (B) at ($(O) + (1.8, 0)$) node at (B) [below] {$\pi/T$};
        \coordinate (C) at ($(O)!1!180:(B)$) node at (C) [below] {$-\pi/T$};
        \coordinate (D) at ($(O)!2!(B)$) node at (D) [below] {$2\pi/T$};
        \coordinate (E) at ($(O)!2!(C)$) node at (E) [below] {$-2\pi/T$};
        \draw [->] (A) -- (O) --++(0:6) node [right] {$\omega$};
        \draw [->] (O) --++(90:6) node [above] {$X(\omega)$};
        \draw (B) --++(90:0.2);
        \draw (C) --++(90:0.2);
        \draw (D) --++(90:0.2);
        \draw (E) --++(90:0.2);
        % \norm{0}{3}{40}{domain=-2:2}{0}
        \norm{3.6}{3}{2}{black}{40}
        \norm{-3.6}{3}{2}{black}{40}
        \norm{0}{3}{2}{black}{40}
      \end{tikzpicture}
    \end{figure}
    後はわからん
  \end{enumerate}
\end{enumerate}
\newpage
\section{}
\subsection{問題文}
\newpage
\subsection{解答}
\begin{enumerate}[(1)]
  \item 
  \begin{eqnarray*}
    (左辺)
    &=& 
    \overline{\overline{x\cdot (\overline{x}\vee\overline{y})}\cdot\overline{y\cdot (\overline{x}\vee \overline{y})}}\\
    &=& 
    \overline{\overline{x\cdot\overline{y}}\cdot\overline{y\cdot\overline{x}}}\\
    &=& 
    \overline{(\overline{x}\vee y)\cdot(\overline{y}\vee x)}\\
    &=& 
    \overline{\overline{x}\cdot\overline{y}\vee x\cdot y}\\
    &=& 
    (x\vee y)\cdot(\overline{x}\vee\overline{y})\\
    &=& 
    x\cdot\overline{y}\vee y\cdot \overline{x}\\
    &=&(右辺)
  \end{eqnarray*}
  よって題意は示された。
  \item notに関して以下が成り立つ。
  \begin{eqnarray*}
     \overline{x} = \overline{x\cdot x}
  \end{eqnarray*}
  よって、notゲートは次のようになる。
  \begin{figure}[H]
    \centering
    \begin{tikzpicture}[>=stealth, circuit logic US]
      \draw (0, 0) node [nand gate] (nand1) {};
      \coordinate (x) at ($(nand1.input 1) + (-1, 0)$);
      \coordinate (y) at ($(x) + (0.5, 0)$);
      \draw (nand1.input 1) -- (x) node [left] {$x$};
      \draw (nand1.input 2) -| (y) node [branch] {};
      \draw (nand1.output) --++(0:0.5) node [right] {$z$};
    \end{tikzpicture}
  \end{figure}
  orに関して以下が成り立つ。
  \begin{eqnarray*}
     x\vee y = \overline{\overline{x\vee y}} = \overline{\overline{x}\cdot \overline{y}}
  \end{eqnarray*}
  よって、orゲートは次のようになる。
  \begin{figure}[H]
    \centering
    \begin{tikzpicture}[>=stealth, circuit logic US]
      \draw (0, 0) node [nand gate] (nand1) {};
      \draw (0, -1) node [nand gate] (nand2) {};
      \draw ($(nand1.output)!0.5!(nand2.output) + (1, 0)$) node [nand gate] (nand3) {};
      \coordinate (x) at ($(nand1.input 1) + (-1, 0)$);
      \coordinate (x') at ($(x) + (0.5, 0)$);
      \draw (nand1.input 1) -- (x) node [left] {$x$};
      \draw (nand1.input 2) -| (x') node [branch] {};
      \coordinate (y) at ($(nand2.input 1) + (-1, 0)$);
      \coordinate (y') at ($(y) + (0.5, 0)$);
      \draw (nand2.input 1) -- (y) node [left] {$y$};
      \draw (nand2.input 2) -| (y') node [branch] {};
      \draw (nand1.output) --++(0:0.3) |- (nand3.input 1);
      \draw (nand2.output) --++(0:0.3) |- (nand3.input 2);
      \draw (nand3.output) --++(0:0.5) node [right] {$z$};
    \end{tikzpicture}
  \end{figure}
  andに関して以下が成り立つ。
  \begin{eqnarray*}
     x\cdot y = \overline{\overline{x\cdot y}}
  \end{eqnarray*}
  andゲートは次のようになる。
  \begin{figure}[H]
    \centering
    \begin{tikzpicture}[>=stealth, circuit logic US]
      \draw (0, 0) node [nand gate] (nand1) {};
      \draw ($(nand1.output) + (2, -0.5)$) node [nand gate] (nand2) {};
      \coordinate (x) at ($(nand1.input 1) + (-1, 0)$);
      \coordinate (y) at ($(nand1.input 2) + (-1, 0)$);
      \coordinate (y') at ($(nand2.input 1) + (-0.8, 0)$);
      \draw (nand1.input 1) -- (x) node [left] {$x$};
      \draw (nand1.input 2) -- (y) node [left] {$y$};
      \draw (nand1.output) --++(0:0.3) |- (nand2.input 1);
      \draw (nand2.input 2) -| (y') node [branch] {};
      \draw (nand2.output) --++(0:0.5) node [right] {$z$};
    \end{tikzpicture}
  \end{figure}
  xorに関して以下が成り立つ。
  \begin{eqnarray*}
     (x\cdot \overline{y})\vee (y\cdot \overline{x}) 
     &=& (x\cdot \overline{x})\vee (x\cdot \overline{y}) \vee 
     (\overline{x}\cdot y) \vee (y\cdot \overline{y})
     = (x\cdot(\overline{x}\vee\overline{y}))\vee(y\cdot(\overline{x}\vee\overline{y}))
     = (x\cdot\overline{x\cdot y})\vee(y\cdot\overline{x\cdot y})\\
     &=& \overline{\overline{(x\cdot\overline{x\cdot y})\vee(y\cdot\overline{x\cdot y})}}
     = \overline{\overline{(x\cdot\overline{x\cdot y})}\cdot\overline{(y\cdot\overline{x\cdot y})}}
  \end{eqnarray*}
  よって、xorゲートは次のようになる。
  \begin{figure}[H]
    \centering
    \begin{tikzpicture}[>=stealth, circuit logic US]
      \draw (0, 0) node [nand gate] (nand1) {};
      \draw ($(nand1.output) + (1.5, 0.5)$) node [nand gate] (nand2) {};
      \draw ($(nand1.output -| nand2)!1!180:(nand2)$) node [nand gate] (nand3) {};
      \draw ($(nand1 -| nand2)!1!180:(nand1)$) node [nand gate] (nand4) {};
      \coordinate (x) at ($(nand1.input 1) + (-1, 0)$);
      \coordinate (y) at ($(nand1.input 2) + (-1, 0)$);
      \draw (nand1.input 1) -- (x) node [left] {$x$};
      \draw (nand1.input 2) -- (y) node [left] {$y$};
      \draw (nand2.input 1) -| ($(x) + (0.5, 0)$) node [branch] {};
      \draw (nand2.input 2) -| ($(nand1.output) + (0.3, 0)$) node [branch] {} -- (nand1.output);
      \draw (nand3.input 2) -| ($(y) + (0.5, 0)$) node [branch] {};
      \draw (nand3.input 1) -| ($(nand1.output) + (0.3, 0)$) node [branch] {} -- (nand1.output);
      \draw (nand2.output) --++(0:0.3) |- (nand4.input 1);
      \draw (nand3.output) --++(0:0.3) |- (nand4.input 2);
      \draw (nand4.output) --++(0:0.5) node [right] {$z$};
    \end{tikzpicture}
  \end{figure}
  \item 
  \begin{enumerate}[(a)]
    \item クロックサイクルと出力$z$の波形は以下のようになる。
    \begin{figure}[H]
      \centering
      \begin{tikzpicture}[>=stealth]
        \coordinate (O) at (0, 0);
        %%%==================================================
        %%破線の列
        \newcount\x
        \x = 0
        \foreach \t in {1, 2, ..., 13}
        {
          \advance\x\t
          \advance\x-1
          \multiply\x1
          \draw ($(O) + (\x, 0.5)$) --++(0:0.5) --++(-90:0.5);
          \ifnum \t < 13
          {
            \node at ($(O) + (\x + 0.5, 0.8)$) {$C_{\t}$};
            \draw ($(O) + (\x + 0.5, 0)$) --++(0:0.5);
          }\fi
          \draw [dotted] ($(O) + (\x, 1)$) --++(-90:3);
          \draw [->] ($(O) + (\x, 0)$) --++(90:0.5);
        }
        %%%==================================================
        \coordinate (c) at ($(O) + (-0.5, 0)$) node at (c) [left] {clock};
        \draw (c) -- (O);
        \coordinate (z) at ($(c) + (0, -1.2)$) node at (z) [left] {z};
        \draw (z) --++(0:0.5);
        \draw (z)++(0:0.5) --++(0:1) --++(90:0.5) --++(0:2) --++(-90:0.5)%
         --++(0:2) --++(90:0.5) --++(0:1) --++(-90:0.5) --++(0:2) --++(90:0.5)%
         --++(0:2) --++(-90:0.5) --++(0:2.5);
      \end{tikzpicture}
    \end{figure}
    \item この送信回路は送信データに1ビット誤り検出用に検査符号を加えたデータを作り、送信するようになっている。
    検査符号は送信データと検査符号を合わせたデータの中で1の数が偶数になるように計算されているため、
    受信した時にデータの中の1の数が偶数にならなければ誤りがあることが分かるようになっているという仕組み。
  \end{enumerate}
  \item 
  \begin{enumerate}[(a)]
    \item way-1の状態は以下のようになる。 
    \begin{table}[H]
      \centering
      \begin{tabular}{cccccc}
        $index$ & $v$ & $tag$ & $lru$ & $data1$ & $data2$\\
        \cline{2-6}
        0 & \multicolumn{1}{|c|}{0} & \multicolumn{1}{|c|}{0} & \multicolumn{1}{|c|}{1} & \multicolumn{1}{|c|}{0} & \multicolumn{1}{|c|}{0}\\
        \cline{2-6}
        1 & \multicolumn{1}{|c|}{1} & \multicolumn{1}{|c|}{3} & \multicolumn{1}{|c|}{2} & \multicolumn{1}{|c|}{(26)} & \multicolumn{1}{|c|}{(27)}\\
        \cline{2-6}
        2 & \multicolumn{1}{|c|}{1} & \multicolumn{1}{|c|}{7} & \multicolumn{1}{|c|}{3} & \multicolumn{1}{|c|}{(30)} & \multicolumn{1}{|c|}{(31)}\\
        \cline{2-6}
        3 & \multicolumn{1}{|c|}{0} & \multicolumn{1}{|c|}{0} & \multicolumn{1}{|c|}{1} & \multicolumn{1}{|c|}{0} & \multicolumn{1}{|c|}{0}\\
        \cline{2-6}
        &\multicolumn{5}{c}{$way$-$1$}
      \end{tabular}
    \end{table}
    \item 以下のようになる。
    \begin{table}[H]
      \centering
      \begin{tabular}{cc}
        \begin{tabular}{c|c}
          \hline
          61 & ヒット\\
          19 & ミス\\
          50 & ヒット\\
          \hline
        \end{tabular}
        \begin{tabular}{cccccc}
          $index$ & $v$ & $tag$ & $lru$ & $data1$ & $data2$\\
          \cline{2-6}
          0 & \multicolumn{1}{|c|}{0} & \multicolumn{1}{|c|}{0} & \multicolumn{1}{|c|}{2} & \multicolumn{1}{|c|}{0} & \multicolumn{1}{|c|}{0}\\
          \cline{2-6}
          1 & \multicolumn{1}{|c|}{1} & \multicolumn{1}{|c|}{6} & \multicolumn{1}{|c|}{3} & \multicolumn{1}{|c|}{(50)} & \multicolumn{1}{|c|}{(51)}\\
          \cline{2-6}
          2 & \multicolumn{1}{|c|}{0} & \multicolumn{1}{|c|}{0} & \multicolumn{1}{|c|}{2} & \multicolumn{1}{|c|}{0} & \multicolumn{1}{|c|}{0}\\
          \cline{2-6}
          3 & \multicolumn{1}{|c|}{0} & \multicolumn{1}{|c|}{0} & \multicolumn{1}{|c|}{2} & \multicolumn{1}{|c|}{0} & \multicolumn{1}{|c|}{0}\\
          \cline{2-6}
          &\multicolumn{5}{c}{$way$-$2$}
        \end{tabular}
      \end{tabular}
    \end{table}
    \item 以下のようになる。
    \begin{table}[H]
      \centering
      \begin{tabular}{c|c}
        \hline
        18 & ヒット\\
        42 & ミス\\
        43 & ヒット\\
        12 & ミス\\
        \hline
      \end{tabular}
    \end{table}
    \begin{table}[H]
      \centering
      \scalebox{0.85}{%
      \begin{tabular}{ccc}
        \begin{tabular}{cccccc}
          $index$ & $v$ & $tag$ & $lru$ & $data1$ & $data2$\\
          \cline{2-6}
          0 & \multicolumn{1}{|c|}{0} & \multicolumn{1}{|c|}{0} & \multicolumn{1}{|c|}{1} & \multicolumn{1}{|c|}{0} & \multicolumn{1}{|c|}{0}\\
          \cline{2-6}
          1 & \multicolumn{1}{|c|}{1} & \multicolumn{1}{|c|}{5} & \multicolumn{1}{|c|}{3} & \multicolumn{1}{|c|}{(42)} & \multicolumn{1}{|c|}{(43)}\\
          \cline{2-6}
          2 & \multicolumn{1}{|c|}{1} & \multicolumn{1}{|c|}{7} & \multicolumn{1}{|c|}{2} & \multicolumn{1}{|c|}{(30)} & \multicolumn{1}{|c|}{(31)}\\
          \cline{2-6}
          3 & \multicolumn{1}{|c|}{0} & \multicolumn{1}{|c|}{0} & \multicolumn{1}{|c|}{1} & \multicolumn{1}{|c|}{0} & \multicolumn{1}{|c|}{0}\\
          \cline{2-6}
          &\multicolumn{5}{c}{$way$-$1$}
        \end{tabular}
        & 
        \begin{tabular}{cccccc}
          $index$ & $v$ & $tag$ & $lru$ & $data1$ & $data2$\\
          \cline{2-6}
          0 & \multicolumn{1}{|c|}{0} & \multicolumn{1}{|c|}{0} & \multicolumn{1}{|c|}{2} & \multicolumn{1}{|c|}{0} & \multicolumn{1}{|c|}{0}\\
          \cline{2-6}
          1 & \multicolumn{1}{|c|}{1} & \multicolumn{1}{|c|}{6} & \multicolumn{1}{|c|}{1} & \multicolumn{1}{|c|}{(50)} & \multicolumn{1}{|c|}{(51)}\\
          \cline{2-6}
          2 & \multicolumn{1}{|c|}{1} & \multicolumn{1}{|c|}{1} & \multicolumn{1}{|c|}{3} & \multicolumn{1}{|c|}{(12)} & \multicolumn{1}{|c|}{(13)}\\
          \cline{2-6}
          3 & \multicolumn{1}{|c|}{0} & \multicolumn{1}{|c|}{0} & \multicolumn{1}{|c|}{2} & \multicolumn{1}{|c|}{0} & \multicolumn{1}{|c|}{0}\\
          \cline{2-6}
          &\multicolumn{5}{c}{$way$-$2$}
        \end{tabular}
        & 
        \begin{tabular}{cccccc}
          $index$ & $v$ & $tag$ & $lru$ & $data1$ & $data2$\\
          \cline{2-6}
          0 & \multicolumn{1}{|c|}{0} & \multicolumn{1}{|c|}{0} & \multicolumn{1}{|c|}{3} & \multicolumn{1}{|c|}{0} & \multicolumn{1}{|c|}{0}\\
          \cline{2-6}
          1 & \multicolumn{1}{|c|}{1} & \multicolumn{1}{|c|}{2} & \multicolumn{1}{|c|}{2} & \multicolumn{1}{|c|}{(18)} & \multicolumn{1}{|c|}{(19)}\\
          \cline{2-6}
          2 & \multicolumn{1}{|c|}{0} & \multicolumn{1}{|c|}{0} & \multicolumn{1}{|c|}{1} & \multicolumn{1}{|c|}{0} & \multicolumn{1}{|c|}{0}\\
          \cline{2-6}
          3 & \multicolumn{1}{|c|}{0} & \multicolumn{1}{|c|}{0} & \multicolumn{1}{|c|}{3} & \multicolumn{1}{|c|}{0} & \multicolumn{1}{|c|}{0}\\
          \cline{2-6}
          &\multicolumn{5}{c}{$way$-$3$}
        \end{tabular}
      \end{tabular}%
      }
    \end{table}
  \end{enumerate}
\end{enumerate}
\index{ティック@tikz}
% \printindex
\end{document}
